\documentclass[11pt]{article}

\usepackage{amsmath,amssymb,amsthm,amsfonts,amscd,tikz} %usual symbols, theorems, etc
\usepackage{tikz-cd} %diagrams
\usepackage{multicol} %columns
\usepackage{enumitem} %makes description more flexible
\usepackage{hyperref} %links, references
\usepackage{aliascnt} %copies counters but with new labels --- for autoref
\usepackage{graphicx} %pictures
\usepackage{subcaption} %subfigures - use minipage
\usetikzlibrary{arrows,matrix,calc,decorations.pathmorphing} %curved arrows
\usepackage{mathabx} %more symbols
\usepackage{mathrsfs} %super curly letters
\usepackage{bbm} %quaternions
\usepackage{microtype} %supposedly makes things nicer, ask Peter
\usepackage[margin=1in]{geometry} %margins
\usepackage{wrapfig} %wraps figures/tables in text
\usepackage{cite} %citations
%\usepackage{sage} %new! sage tex package


%GENERAL SYMBOLS
\newcommand{\BB}[1]{\mathbb{#1}} %bold face
\newcommand{\script}[1]{\mathcal{#1}} %curvy
\newcommand{\curly}[1]{\mathscr{#1}} %extra curvy
\newcommand{\cat}[1]{\textbf{\emph{#1}}} %bold
\renewcommand{\frak}[1]{\mathfrak{#1}} %gothic
\newcommand{\del}[1]{\frac{\partial}{\partial{#1}}} %single derivative vector d/dx
\newcommand{\ddel}[2]{\frac{\partial{#1}}{\partial{#2}}} %dy/dx
\newcommand{\floor}[1]{{\lfloor #1 \rfloor}} %flooor
\newcommand{\free}[1]{\left\langle#1\right\rangle} %nice brackets

%SPECIFIC SYMBOLS
\newcommand{\CC}{\BB{C}}
\newcommand{\RR}{\BB{R}}
\newcommand{\NN}{\BB{N}}
\newcommand{\QQ}{\BB{Q}}
\newcommand{\ZZ}{\BB{Z}}
\newcommand{\PP}{\BB{P}}
\renewcommand{\AA}{\BB{A}} %\AA is a non-ascii acented A, weird
\newcommand{\HH}{\BB{H}}
\newcommand{\DD}{\BB{D}}
\newcommand{\TT}{\BB{T}}
\newcommand{\FF}{\BB{F}}
\newcommand{\RP}{\RR\PP}
\newcommand{\CP}{\CC\PP}
\renewcommand{\SS}{\BB{S}} %\SS is a silly non-ascii letter "SS".
\newcommand{\sF}{\script{F}}
\newcommand{\sL}{\script{L}}
\newcommand{\sG}{\script{G}}
\renewcommand{\O}{\script{O}}
\newcommand{\st}{\mid}
\renewcommand{\div}{\operatorname{div}} %\div is a silly '%'
\newcommand{\im}{\operatorname{Im}}
\newcommand{\re}{\operatorname{Re}}
\newcommand{\rank}{\operatorname{rank}}
\newcommand{\codim}{\operatorname{codim}}
\newcommand{\diag}{\operatorname{diag}}
\newcommand{\supp}{\operatorname{Supp}}
\newcommand{\Aut}{\operatorname{Aut}}
\newcommand{\End}{\operatorname{End}}
\newcommand{\tr}{\operatorname{tr}}
\newcommand{\GL}{\operatorname{GL}}
\newcommand{\SL}{\operatorname{SL}}
\newcommand{\Pic}{\operatorname{Pic}}
\newcommand{\Gal}{\operatorname{Gal}}
\newcommand{\chr}{\operatorname{char}}
\newcommand{\ord}{\operatorname{ord}}
\newcommand{\res}{\operatorname{res}}
\newcommand{\Hom}{\operatorname{Hom}}
\renewcommand{\hat}{\widehat}

%counter for things
\newtheorem{counter}{plzplzplzplzplzplzdontuseme}[section]

\theoremstyle{plain}

\newaliascnt{theoremCounter}{counter}
\newtheorem{thm}[theoremCounter]{Theorem}
\aliascntresetthe{theoremCounter}
\newcommand{\theoremCounterautorefname}{Theorem}
\newaliascnt{propositionCounter}{counter}
\newcommand{\propositionCounterautorefname}{Proposition}
\newtheorem{prop}[propositionCounter]{Proposition}
\aliascntresetthe{propositionCounter}
\newaliascnt{lemmaCounter}{counter}
\newcommand{\lemmaCounterautorefname}{Lemma}
\newtheorem{lem}[lemmaCounter]{Lemma}
\aliascntresetthe{lemmaCounter}
\newaliascnt{corollaryCounter}{counter}
\newcommand{\corollaryCounterautorefname}{Corollary}
\newtheorem{cor}[corollaryCounter]{Corollary}
\aliascntresetthe{corollaryCounter}


\theoremstyle{definition}

\newaliascnt{definitionCounter}{counter}
\newtheorem{defn}[definitionCounter]{Definition}
\aliascntresetthe{definitionCounter}
\newcommand{\definitionCounterautorefname}{Definition}
\newaliascnt{exampleCounter}{counter}
\newtheorem{ex}[exampleCounter]{Example}
\aliascntresetthe{exampleCounter}
\newcommand{\exampleCounterautorefname}{Example}
\newaliascnt{exerciseCounter}{counter}
\newtheorem{excer}[exerciseCounter]{Exercise}
\aliascntresetthe{exerciseCounter}
\newcommand{\exerciseCounterautorefname}{Exercise}
\newaliascnt{warniningCounter}{counter}
\newtheorem{warn}[warniningCounter]{Warning}
\aliascntresetthe{warniningCounter}
\newcommand{\warniningCounterautorefname}{Warning}
\newaliascnt{factCounter}{counter}
\newtheorem{fact}[factCounter]{Fact}
\aliascntresetthe{factCounter}
\newcommand{\factCounterautorefname}{Fact}

\theoremstyle{remark}

\newaliascnt{remarkCounter}{counter}
\newtheorem{rem}[remarkCounter]{Remark}
\aliascntresetthe{remarkCounter}
\newcommand{\remarkCounterautorefname}{Remark}
\newtheorem*{claim}{Claim}

\newtheorem*{hint}{Hint}


%page style, title page, custom variables
\newcommand{\mytitle}{Riemann-Roch Theorem}
\newcommand{\myauthor}{Travis Scholl}
\newcommand{\myemail}{tscholl2@uw.edu}
\date{April 23, 2015}


\begin{document}
\title{\bfseries\sffamily \mytitle}
\author{\sc \myauthor }
\maketitle

\begin{abstract}
	*These notes were taken from a course by Ralph Greenburg in Spring 2015 \cite{ralphscourse} on counting points on varieties over finite fields. As usual, any mistakes should be due to me.
\end{abstract}

\section{Motivation: Riemann Manifolds}
\hfill

Let $X$ be a connected, compact Riemann surface, $f$ a non-zero meromorphic function on $X$, and $P$ a point in $X$.

\begin{defn}
	Let $\ord_P f$ be the \emph{order vanishing of $f$ at $P$}. In particular,
	$$
	\ord_P f =
	\begin{cases}
		\text{order vanishing} &\text{if $f$ is holomorphic at $P$}
		\\
		-(\text{order of the pole}) &\text{if $f$ has a pole at $P$}
	\end{cases}
	$$
\end{defn}

Recall the standard fact from the theory of compact Riemann surfaces.

\begin{fact}\label{divffinitedegree}
	If $f\neq 0$ then $\ord_P f = 0$ for all but finitely many points $P$.
\end{fact}

Next we define divisors and the divisor class group for $X$.

\begin{defn}\label{divisors}
	The \emph{divisors} of $X$ are formal $\ZZ$-linear combinations of points, i.e. $\sum_{P\in X}a_P P$ for some $a_P\in\ZZ$ and for all but finitely many $P$ we have $a_P=0$.

	Given a divisor $D = \sum_{P\in X}a_P P$ define the \emph{degree} of $D$ to be $\deg(D) = \sum_{P\in X} a_P$. This is finite by definition.

	For a non-zero meromorphic function $f$ define the \emph{divisor of $f$} to be
	$$
	\div f = \sum_{P\in X} (\ord_P f)P.
	$$
	Divisors of the form $\div f$ for some $f$ are called \emph{principle divisors}.
\end{defn}

Another useful fact from general compact Riemann surfaces will allow us to define the class group.

\begin{fact}\label{degdivf=0}
	If $f$ is a non-zero meromorphic function on $X$, then $\deg(\div f) = 0$, i.e. the number of zeros equals the number of poles counting multiplicity.
\end{fact}

\begin{defn}\label{classgp}
	The \emph{divisor class group} of $X$ is
	$$
	\frac{\text{Group of degree $0$ divisors}}{\text{Group of principle divisors}}.
	$$
\end{defn}

Note that the divisor class group is well defined by \autoref{degdivf=0}. Next we want to use divisors to count classes of functions.

\begin{defn}
	Let $D = \sum_{P\in X}a_P$ be a divisor. We say $D$ is \emph{effective} if $D \geq 0$ meaning $a_P\geq 0$ for all $P$. Then define
	$$
	\sL(D) = \left\{ f \st \div f + D \text{ is effective} \right\}\cup\{0\}.
	$$
	By \autoref{sLvectorspace} $\sL(D)$ is a vector space over $\CC$. Define $\ell(D)$ to be $\dim_{\CC}\sL(D)$.
\end{defn}

\begin{excer}\label{sLvectorspace}
	Show that $\sL(D)$ is a complex vector space under the usual scaling action on functions. The work will be in showing that it is closed under addition. This requires some basic complex analysis.
	\begin{hint}
		Consider what happens to the poles and zeros of functions when you add them. Use the power series representations since you only have to look locally.
	\end{hint}
\end{excer}

\begin{fact}\label{falwayshasazero}
	A non-constant meromorphic function $f$ on $X$ has at least one zero and one pole.
\end{fact}

\begin{ex}\label{key}
	Let $D_0$ be the zero divisor, i.e. $D_0 = \sum_{P\in X}0\cdot P$. By \autoref{falwayshasazero}, $\sL(D_0)$ consists of only constant functions so $\ell(D_0) = 1$.
\end{ex}

\begin{prop}
	If $\deg D = 0$ then $\sL(D) = \{f \st \div f = -D\}\cup\{0\}$ and $\ell(D)$ is either $0$ or $1$.
\end{prop}
\begin{proof}
	Since $\deg D = 0$ we have by definition $\sL(D) = \{f \st \div f \geq -D\}\cup\{0\}$. If $\div f > -D$ then $\deg \div f > \deg - D = 0$, contradicting \autoref{degdivf=0}. The second statement follows from the fact that given two meromorphic functions $f,g$ with $\div f = \div g$ then $\frac{f}{g}$ is a well defined meromorphic function on $X$ with no zeros or poles. Hence it must be constant and so $\ell(D) \leq 1$.
\end{proof}

\begin{rem}
	If $X$ has positive genus, then $\ell(D)$ is rarely $1$ in the previous proposition. If $X$ has genus $0$ then $\ell(D) = 1$ for every divisor of degree $0$. This follows from the fact that the divisor class group is trivial. The idea is that $X \cong \PP^1(\CC)$, the Riemann sphere. Meromorphic functions on $\PP^1(\CC)$ are just rational polynomials and we can describe any finite set of zeros and poles.
\end{rem}

\bibliographystyle{alpha}
\bibliography{references}


\begin{center}
\noindent\rule{4cm}{.5pt}
\vspace{.25cm}

\noindent {\sc \small \myauthor}\\
{\small Department of Mathematics, University of Washington, Seattle WA 98195} \\
email: {\tt \myemail}
\end{center}

\end{document}
